%\documentclass[a4paper,landscape,twocolumn,12pt]{article}
\documentclass[a4paper,12pt]{article}

\usepackage[french]{babel}
\usepackage[utf8]{inputenc}
\usepackage[T1]{fontenc}
\usepackage{graphicx}
\usepackage{tabularx}
\usepackage{color}
\usepackage{tikz}\usetikzlibrary{shapes.geometric}
\usepackage{url}
\usepackage{import,palatino}
\usepackage{pdfpages,wrapfig,setspace}
\usepackage[margin=15mm]{geometry}

\setlength{\parskip}{\smallskipamount}
%\setlength{\parindent}{0pt}

\pagestyle{empty}
\begin{document}
\begin{center}
  {\Huge Ceci est un petit livre}

  {\Large à construire vous-même}
\end{center}

\large

Vous trouverez de ce coté les instructions de fabrication de votre
petit livre, qui se trouve de l'autre coté de la feuille.  Pas besoin
de colle, juste de ciseaux: 

\bigskip\bigskip


\noindent\textbf{Étape 1:} Pliez votre feuille en deux.
Les bords doivent être bien jointifs.

\noindent
\begin{minipage}[b]{.45\linewidth}
\medskip
\noindent\textbf{Étape 2:} Repliez encore en deux, puis encore en deux.  Les plis doivent être
bien marqués.

\medskip \centerline{\includegraphics{img/ptitlivre-etape2.jpg}}

\medskip
\noindent\textbf{Étape 3:} Découpez le pli au milieu de la page.

\medskip \centerline{\includegraphics{img/ptitlivre-etape3.jpg}}

\end{minipage}\hfill\begin{minipage}[b]{.45\linewidth}
\noindent\textbf{Étape 4:} Repliez la feuille dans le sens de la longueur.\\

  \centerline{\includegraphics{img/ptitlivre-etape4.jpg}}
  
\medskip
\noindent\textbf{Étape 5:} Repliez les deux parties centrales, repliez
le tout et c'est fini!

  \centerline{\includegraphics{img/ptitlivre-etape5.jpg}}

\end{minipage}

\bigskip~\hfill{\small Le concept du «petit livre» est une idée
  originale de {\color{blue}\url{http://petitslivres.free.fr/}}}

\bigskip \bigskip \bigskip %
Ce petit livre, créé par Martin Quinson et Jean-Christophe Bach, est
diffusé en CC-BY-SA. Merci de nous indiquer toute erreur ou
amélioration possible sur:

\centerline{\color{blue}\url{https://github.com/jcb/CSIRL}}

\bigskip%
Il s'agit de la version du 4 février 2014, la première connue.
\end{document}
