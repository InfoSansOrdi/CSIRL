\begin{frame}{Activité: Le crêpier psycho-rigide}

  \begin{block}{Matériel}
    \begin{itemize}
    \item des planchettes en bois de tailles et de couleurs différentes (faces reconnaissables)
    \item éventuellement une pelle à tarte pour retourner les planchettes
    \end{itemize}
  \end{block}

  \begin{block}{Règle du jeu}
    \begin{itemize}
      \item \structure{Installation :} Faire une pile désordonnée de crêpes.
      \item \structure{Objectif :} ranger les crêpes de la plus grande (en bas) à la plus petite (au haut), face colorée vers le haut.
      \item \structure{Coup autorisé :} prendre une ou plusieurs crêpes sur le haut de la pile, et de les reposer à l'envers.
    \end{itemize}
  \end{block}

  \bigskip \bigskip \bigskip

  \begin{center}
    \includegraphics[width=0.8\linewidth]{img/crepier.pdf}
  \end{center}

\end{frame}

\begin{frame}{Ce qu'il faut retenir du  crêpier psycho-rigide}

  \begin{columns}
    \begin{column}{.7\linewidth}
      \begin{block}{Un algorithme}
        \begin{itemize}
        \item n'a d'intérêt que si on peut l'expliquer
        \item doit être suffisamment simple pour pouvoir l'expliquer à une machine
        \item \alert{\textbf{«Diviser pour mieux régner»}} : on essaie toujours de décomposer un algorithme en tâches simples
        \end{itemize}
      \end{block}

      \begin{block}{L'algorithme que doit suivre le crêpier est :}
        \begin{itemize}
        \item ramener la plus grande crêpe en haut de la pile
        \item retourner pour que la face brûlée soit vers le haut
        \item retourner la pile de sorte à mettre la plus grande crêpe en bas
        \item réitérer avec la crêpe de taille inférieure
        \end{itemize}
      \end{block}

      \begin{block}{Le rapport avec l'informatique}
        \begin{itemize}
        \item l'informaticien passe son temps à trouver des algorithmes et  à les expliquer à la machine
        \item le principe \alert{\textbf{«Diviser pour mieux régner»}} est fondamental en informatique
        \end{itemize}
      \end{block}
    \end{column}

    \begin{column}{.3\linewidth}
      \begin{block}{Pour aller plus loin}
        Selon l'état initial de la pile de crêpes, le nombre minimum de coups nécessaires pour la ranger varie.

        \begin{itemize}
          \item Quel est le meilleur état initial possible (qui demandera le moins de coups pour ranger) ?
          \item Quel est le pire état initial possible ?
          \item Combien faut-il de coups pour ranger une pile de $N$ crêpes dans le pire des cas ?
        \end{itemize}
      \end{block}
    \end{column}
  \end{columns}
      
\end{frame}
