\newcommand{\maisonPair}[5]{ \begin{tikzpicture}
  \node[name=m,shape=regular polygon,regular polygon sides=#3,minimum size=22mm,rotate=(360/#3)]{};
  \node[name=b,shape=regular polygon,regular polygon sides=#4,minimum size=14mm,rotate=(360/#4)/2]{};
  \foreach \base/\maison in {#5} {
    \draw[shift=(m.corner \base)]
       node[shape=ellipse,fill=\maison,draw=black,rotate=((360/#3)*(\base-1))+(360/#3/2)] {~~};
  }
  \foreach \bb in {1,...,#4} {
    \draw[shift=(b.corner \bb)] node[name=bb \bb]{};
  }
  \foreach \base/\maison in {#1} {
    \draw[shift=(b.corner \base)]
       node[name=bb \base,shape=circle,fill=\maison,draw=black,inner sep=.1]
       {~~~};
%         {\footnotesize\base};
  }
  #2
\end{tikzpicture} }
\newcommand{\maisonImpair}[5]{ \begin{tikzpicture}
  \node[name=m,shape=regular polygon,regular polygon sides=#3,
        minimum size=22mm, inner sep=0pt]{};
  \node[name=b,shape=regular polygon,regular polygon sides=#4,minimum size=14mm]{};
  \foreach \base/\maison in {#5} {
    \draw[shift=(m.corner \base)]
       node[shape=ellipse,fill=\maison,draw=black,rotate=(360/#3)*(\base-1)] {~~};
  }
  \foreach \base/\maison in {#1} {
    \draw[shift=(b.corner \base)]
       node[shape=circle,fill=\maison,draw=black,inner sep=.1] {~~~};
  }
  \foreach \bb in {1,...,#4} {\draw[shift=(b.corner \bb)] node[name=bb \bb] {};}
%  \foreach \bb in {1,...,#4} {\draw[shift=(b.corner \bb)] node[name=bb \bb]{\tiny\bb};}%debug the bonshommes names
  #2
\end{tikzpicture} }

\newcommand{\maisonQuatre}[2]{\maisonPair{#1}{#2}{4}{12}{1/A,2/B,3/C,4/D}}
\newcommand{\maisonCinq}[2]{\maisonImpair{#1}{#2}{5}{20}{1/A,2/B,3/C,4/D,5/E}}
\newcommand{\maisonSix}[2]{\maisonPair{#1}{#2}{6}{24}{1/A,2/B,3/C,4/D,5/E,6/F}}
\newcommand{\maisonSept}[2]{\maisonImpair{#1}{#2}{7}{28}{1/A,2/B,3/C,4/D,5/E,6/F,7/G}}

\colorlet{A}{green!60}
\colorlet{B}{red!80}
\colorlet{C}{purple!40}
\colorlet{D}{black!2!yellow}
\colorlet{E}{blue!70}
\colorlet{F}{orange!80}
\colorlet{G}{olive}
\colorlet{H}{magenta}
\colorlet{I}{lime}
\colorlet{J}{pink}

\newcommand{\pawn}[1]{\tikz \draw node[shape=circle,fill=#1,draw=black,inner sep=.1] {~~~};}



%%%%%%%%%%%%%%%%%%%%%%%%%%%%%%%%%%%%%%%%%%%%%%%%%%%%%%%%%%%%%%%%%%%%%%%%%%%%%%%%%%%%%%%
\begin{frame}{Activité: Base-ball multicolore}
  \begin{block}{Matériel nécessaire}
    \begin{itemize}
    \item Plusieurs équipes bien différentiables, chacune composée d'une maison
      et de deux bonshommes\\
      (des legos, des bouts de bois, des cailloux, du fil électrique de
      différentes couleurs, ou autres)

    \item Il faut 5 équipes au minimum. 7 équipes sont préférables, et on peut
      en utiliser une dizaine
    \end{itemize}
  \end{block}

  \begin{block}{Règles du jeu (exemple à quatre équipes)}
    \begin{itemize}
    \item \structure{Situation initiale:}~\\
      On place 4 maisons autour du terrain, et on répartit 7 bonshommes au
      hasard dans les maisons\\
     (il manque donc l'un des bonshommes, qui est mis de coté et ne sera
      pas utilisé)
    \item \structure{Coups autorisés:}
      \begin{itemize}\normalsize
      \item On peut déplacer un seul bonhomme à la fois, d'une maison vers une
        maison adjacente\\
        $\leadsto$ il est interdit de traverser le terrain
      \item On ne peut jamais avoir plus de 2 bonshommes par maison\\
        $\leadsto$ on ne peut bouger que d'une maison adjacente vers la maison
        où il y a un trou
      \end{itemize}
    \item  \structure{Situation finale:}
      L'objectif est de ramener tous les bonshommes dans la maison de leur couleur.
    \end{itemize}
  \end{block}

  \vspace{-\baselineskip}
  \begin{columns}
    \begin{column}{.17\linewidth}\center
      \maisonQuatre{2/B,3/A, 5/D, 8/B,9/C, 11/D,12/A} {}

      \structure{Situation initiale}
    \end{column}
    \begin{column}{.17\linewidth}\center
      \maisonQuatre{2/B,3/A, 5/D, 8/B,9/C, 11/D,12/A}
                   {
                     \draw[->,ultra thick,draw=black!10!green] (bb 2) -> (bb 6);
                     \draw[->,ultra thick,draw=black!10!green] (bb 3) -> (bb 6);
                     \draw[->,ultra thick,draw=black!10!green] (bb 9) -> (bb 6);
                     \draw[->,ultra thick,draw=black!10!green] (bb 8) -> (bb 6);}

      \structure{Coups autorisés}
    \end{column}
    \begin{column}{.17\linewidth}\center
      \maisonQuatre{2/B,3/A, 5/D, 8/B,9/C, 11/D,12/A}
                   { \draw[->,ultra thick,draw=red] (bb 11) -> (bb 6);
                     \draw[->,ultra thick,draw=red] (bb 12) -> (bb 6);
                     \draw[->,ultra thick,draw=red] (bb 12) -> (bb 2);
                     \draw[->,ultra thick,draw=red] (bb 2) -> (bb 12);
                     \draw[->,ultra thick,draw=red] (bb 5) -> (bb 3);
                     \draw[->,ultra thick,draw=red] (bb 5)
                        .. controls (0pt,0pt) .. (bb 8);
                     \draw[->,ultra thick,draw=red] (bb 11) -> (bb 9);
                     \draw[->,ultra thick,draw=red] (bb 9) -> (bb 11);
                   }

      \structure{Coups interdits}
    \end{column}
    \begin{column}{.17\linewidth}\center
      \maisonQuatre{2/A,3/A, 5/B,6/B, 8/C,9/C, 11/D}{}

      \structure{Situation finale}
    \end{column}
  \end{columns}

  \bigskip
  \begin{block}{Objectif de l'activité}
    \begin{itemize}
    \item Le plus important n'est pas  de ramener les bonshommes dans leurs
      maisons
    \item L'objectif est d'expliquer comment on fait. On cherche donc
      l'algorithme correspondant
    \item Si on veut utiliser plus de couleurs, c'est possible.
    \end{itemize}
  \end{block}

  \centerline{
    \maisonCinq{1/A,2/A, 5/B,6/B, 9/C,10/C, 13/D,14/D, 17/E}{}
    \maisonSix{2/A,4/A,  6/B,8/B, 10/C,12/C, 14/D,16/D, 18/E,20/E, 22/F}{}
    \maisonSept{1/A,2/A, 5/B,6/B, 9/C,10/C, 13/D,14/D, 17/E,18/E, 21/F,22/F, 25/G}{}
    \maisonPair{2/A,4/A, 6/B,8/B, 10/C,12/C, 14/D,16/D, 18/E,20/E, 22/F,24/F,
      26/G,28/G, 30/H}{}{8}{32}
               {1/A,2/B,3/C,4/D,5/E,6/F,7/G,8/H}
    % Neuf maisons
    \maisonImpair{1/A,2/A, 5/B,6/B, 9/C,10/C, 13/D,14/D, 17/E,18/E,
                  21/F,22/F, 25/G,26/G, 29/H,30/H,  33/I}{}{9}{36}
               {1/A,2/B,3/C,4/D,5/E,6/F,7/G,8/H,9/I}
    % Dix maisons
    \maisonPair{2/A,4/A, 6/B,8/B, 10/C,12/C, 14/D,16/D, 18/E,20/E, 22/F,24/F,
      26/G,28/G, 30/H,32/H, 34/I,36/I, 38/J}{}{10}{40}
               {1/A,2/B,3/C,4/D,5/E,6/F,7/G,8/H,9/I,10/J}
  }


\end{frame}
%%%%%%%%%%%%%%%%%%%%%%%%%%%%%%%%%%%%%%%%%%%%%%%%%%%%%%%%%%%%%%%%%%%%%%%%%%%%%%%%%%%%%%%%%
\newcommand{\flecherond}[1]{
  \draw[ultra thick] (0,0) circle (3mm);
  \draw[ultra thick,rotate=#1*72] (3mm,0) -- +(-.15,-.08);
  \draw[ultra thick,rotate=#1*72] (3mm,0) -- +(.08,-.15);
  \draw[fill=white,draw=white,rotate=#1*72] (3mm,2.5pt) circle (2pt);
}
\begin{frame}{Un premier algorithme pour le base-ball multicolore}
  \begin{block}{L'algorithme}
    \begin{itemize}
    \item On ne s'autorise qu'à tourner dans un seul sens.
    \item On n'a plus 4 coups possibles, mais deux seulement (car 2 bonshommes
      tourneraient à l'envers)
    \item Entre ces deux bonshommes, je déplace celui dont la destination est la
      plus lointaine
    \end{itemize}
  \end{block}


  \begin{block}{Exemple d'exécution}
    \begin{columns}
  % J'avais une case de trop...
  % \begin{column}{.15\linewidth}\center
  %   \maisonCinq{1/D,2/B, 5/A,6/C, 9/E,10/D, 13/A,14/C, 17/B}{\flecherond{0}}\\
  %   {\small $4>2\leadsto$ \pawn{C}}        
  % \end{column}

  \begin{column}{.15\linewidth}\center
    \maisonCinq{1/D,2/B, 5/A,6/C, 9/E,10/D, 13/A, 17/B,18/C}{\flecherond{4}}\\
    {\small $2>1\leadsto$ \pawn{E}}        
  \end{column}

  \begin{column}{.15\linewidth}\center
    \maisonCinq{1/D,2/B, 5/A,6/C, 10/D, 13/A,14/E, 17/B,18/C}{\flecherond{3}}\\
    {\small $4>1\Rightarrow$ \pawn{A}}
  \end{column}

  \begin{column}{.15\linewidth}\center
    \maisonCinq{1/D,2/B, 6/C, 9/A,10/D, 13/A,14/E, 17/B,18/C}{\flecherond{2}}\\
    {\small $3>1\leadsto$ \pawn{D}}        
  \end{column}

  \begin{column}{.15\linewidth}\center
    \maisonCinq{2/B, 5/D,6/C, 9/A,10/D, 13/A,14/E, 17/B,18/C}{\flecherond{1}}\\
    {\small $3>2\leadsto$ \pawn{C}}        
  \end{column}
  \begin{column}{.15\linewidth}\center
    \maisonCinq{1/C,2/B, 5/D,6/C, 9/A,10/D, 13/A,14/E, 17/B}{\flecherond{0}}\\
    {\small $2>1\leadsto$ \pawn{A}}        
  \end{column}
\end{columns}
%%%%%%%%%%%%%
\begin{columns}
  \begin{column}{.15\linewidth}\center
    \maisonCinq{1/C,2/B, 5/D,6/C, 9/A,10/D, 14/E, 17/B,18/A}{\flecherond{4}}\\
    {\small $2>1\leadsto$ \pawn{A}}        
  \end{column}

  \begin{column}{.15\linewidth}\center
    \maisonCinq{1/C,2/B, 5/D,6/C, 10/D, 13/A,14/E, 17/B,18/A}{\flecherond{3}}\\
     {\small $2>1\leadsto$ \pawn{D}}        
  \end{column}

  \begin{column}{.15\linewidth}\center
    \maisonCinq{1/C,2/B, 6/C, 9/D,10/D, 13/A,14/E, 17/B,18/A}{\flecherond{2}}\\
     {\small $2>1\leadsto$ \pawn{C}}        
  \end{column}

  \begin{column}{.15\linewidth}\center
    \maisonCinq{2/B, 5/C,6/C, 9/D,10/D, 13/A,14/E, 17/B,18/A}{\flecherond{1}}\\
     {\small $2>1\leadsto$ \pawn{B}}        
  \end{column}
  \begin{column}{.15\linewidth}\center
    \maisonCinq{1/B,2/B, 5/C,6/C, 9/D,10/D, 13/A,14/E, 18/A}{\flecherond{0}}\\
     {\small $2>1\leadsto$ \pawn{A}}        
  \end{column}
\end{columns}
%%%%%%%%%%%%%
\begin{columns}
  \begin{column}{.15\linewidth}\center
    \maisonCinq{1/B,2/B, 5/C,6/C, 9/D,10/D, 14/E, 17/A,18/A}{\flecherond{4}}\\
     {\small $1=1\leadsto$ \pawn{D}}        
  \end{column}

  \begin{column}{.15\linewidth}\center
    \maisonCinq{1/B,2/B, 5/C,6/C, 10/D, 13/D,14/E, 17/A,18/A}{\flecherond{3}}\\
     {\small $1=1\leadsto$ \pawn{C}}        
  \end{column}

  \begin{column}{.15\linewidth}\center
    \maisonCinq{1/B,2/B, 6/C, 9/C,10/D, 13/D,14/E, 17/A,18/A}{\flecherond{2}}\\
     {\small $1=1\leadsto$ \pawn{B}}        
  \end{column}

  \begin{column}{.15\linewidth}\center
    \maisonCinq{2/B, 5/B,6/C, 9/C,10/D, 13/D,14/E, 17/A,18/A}{\flecherond{1}}\\
     {\small $1=1\leadsto$ \pawn{A}}        
  \end{column}

  \begin{column}{.15\linewidth}\center
    \maisonCinq{1/A,2/B, 5/B,6/C, 9/C,10/D, 13/D,14/E, 18/A}{\flecherond{0}}\\
     {\small $1>0\leadsto$ \pawn{E}}        
  \end{column}
\end{columns}
%%%%%%%%%%%%%
\begin{columns}[t]
  \begin{column}{.15\linewidth}\center
    \maisonCinq{1/A,2/B, 5/B,6/C, 9/C,10/D, 13/D, 17/E,18/A}{\flecherond{4}}\\
     {\small $1>0\leadsto$ \pawn{D}}
  \end{column}

  \begin{column}{.15\linewidth}\center
    \maisonCinq{1/A,2/B, 5/B,6/C, 9/C, 13/D,14/D, 17/E,18/A}{\flecherond{3}}\\
     {\small $1>0\leadsto$ \pawn{C}}
  \end{column}

  \begin{column}{.15\linewidth}\center
    \maisonCinq{1/A,2/B, 5/B, 9/C,10/C, 13/D,14/D, 17/E,18/A}{\flecherond{2}}\\
     {\small $1>0\leadsto$ \pawn{B}}
  \end{column}

  \begin{column}{.15\linewidth}\center
    \maisonCinq{1/A, 5/B,6/B, 9/C,10/C, 13/D,14/D, 17/E,18/A}{\flecherond{1}}\\
     {\small $1>0\leadsto$ \pawn{A}}
  \end{column}

  \begin{column}{.15\linewidth}\center
    \maisonCinq{1/A,2/A, 5/B,6/B, 9/C,10/C, 13/D,14/D, 17/E}{\flecherond{0}}\\
     \alert{Gagné!}
  \end{column}
\end{columns}

  \end{block}
\end{frame}
%%%%%%%%%%%%%%%%%%%%%%%%%%%%%%%%%%%%%%%%%%%%%%%%%%%%%%%%%%%%%%%%%%%%%%%%%%%%%%%%%%%%%%%%%
\newcommand{\maisonOk}[1]{\foreach \x in {#1} {\draw[shift=(m.corner \x)] node
    {\small x};}}
\begin{frame}{Étude du premier algorithme pour le base-ball multicolore}
  \begin{block}{Cet algorithme semble attirant}
    \begin{itemize}
    \item Il est très simple: on pourrait l'expliquer à un ordinateur
    \item Il est relativement rapide: 20 coups pour 9 bonshommes, ce n'est pas
      si mal
    \item Seul problème: cet algorithme est faux: dans certains cas, il ne
      termine jamais\ldots
    \end{itemize}
  \end{block}

  \begin{block}{Exemple d'exécution incorrecte}
    \begin{itemize}
    \item Il suffit d'inverser deux pions sur une situation gagnée (ici,
      \pawn{A} et \pawn{D}), et on applique l'algorithme 
    \end{itemize}\vspace{-\baselineskip}
    \begin{columns}[t]
  \begin{column}{.15\linewidth}\center
    \maisonCinq{1/A,2/D, 5/B,6/B, 9/C,10/C, 13/A,14/D, 18/E}{
      \flecherond{0}\maisonOk{2,3,5} }\\
     {\small $2>0\leadsto$ \pawn{A}}
  \end{column}

  \begin{column}{.15\linewidth}\center
    \maisonCinq{1/A,2/D, 5/B,6/B, 9/C,10/C, 14/D, 17/A,18/E}
      {\flecherond{4}\maisonOk{2,3,5}}\\
     {\small $0=0\leadsto$ \pawn{C}}
  \end{column}

  \begin{column}{.15\linewidth}\center
    \maisonCinq{1/A,2/D, 5/B,6/B, 10/C, 13/C,14/D, 17/A,18/E}
      {\flecherond{3}\maisonOk{2,5}}\\
     {\small $0=0\leadsto$ \pawn{B}}
  \end{column}

  \begin{column}{.15\linewidth}\center
    \maisonCinq{1/A,2/D, 6/B, 9/B,10/C, 13/C,14/D, 17/A,18/E}
      {\flecherond{2}\maisonOk{5}}\\
     {\small $3>0\leadsto$ \pawn{D}}
  \end{column}

  \begin{column}{.15\linewidth}\center
    \maisonCinq{1/A, 5/D,6/B, 9/B,10/C, 13/C,14/D, 17/A,18/E}
      {\flecherond{1}\maisonOk{5}}\\
     {\small $1>0\leadsto$ \pawn{A}}
  \end{column}

  \begin{column}{.15\linewidth}\center
    \maisonCinq{1/A,2/A, 5/D,6/B, 9/B,10/C, 13/C,14/D, 18/E}
      {\flecherond{0}\maisonOk{1,5}}\\
     {\small $4>0\leadsto$ \pawn{C}}
  \end{column}
\end{columns}
%%%%%%%%%%%%%
\begin{columns}

  \begin{column}{.15\linewidth}\center
    \maisonCinq{1/A,2/A, 5/D,6/B, 9/B,10/C, 14/D, 17/C,18/E}
      {\flecherond{4}\maisonOk{1,5}}\\
      {\small $4>0\leadsto$ \pawn{B}}
  \end{column}

  \begin{column}{.15\linewidth}\center
    \maisonCinq{1/A,2/A, 5/D,6/B, 10/C, 13/B,14/D, 17/C,18/E}
      {\flecherond{3}\maisonOk{1,5}}\\
     {\small $2>0\leadsto$ \pawn{D}}
  \end{column}

  \begin{column}{.15\linewidth}\center
    \maisonCinq{1/A,2/A, 6/B, 9/D,10/C, 13/B,14/D, 17/C,18/E}
      {\flecherond{2}\maisonOk{1,5}}\\
     {\small $0=0\leadsto$ \pawn{A}}
  \end{column}

  \begin{column}{.15\linewidth}\center
    \maisonCinq{1/A, 5/A,6/B, 9/D,10/C, 13/B,14/D, 17/C,18/E}
      {\flecherond{1}\maisonOk{5}}\\
     {\small $3>0\leadsto$ \pawn{C}}
  \end{column}

  \begin{column}{.15\linewidth}\center
    \maisonCinq{1/A,2/C, 5/A,6/B, 9/D,10/C, 13/B,14/D, 18/E}
      {\flecherond{0}\maisonOk{5}}\\
     {\small $3>0\leadsto$ \pawn{B}}
  \end{column}

  \begin{column}{.15\linewidth}\center
    \maisonCinq{1/A,2/C, 5/A,6/B, 9/D,10/C, 14/D, 17/B,18/E}
      {\flecherond{4}\maisonOk{5}}\\
      {\small $1>0\leadsto$ \pawn{D}}
  \end{column}
\end{columns}
%%%%%%%%%%%%%
\begin{columns}

  \begin{column}{.15\linewidth}\center
    \maisonCinq{1/A,2/C, 5/A,6/B, 10/C, 13/D,14/D, 17/B,18/E}
      {\flecherond{3}\maisonOk{4,5}}\\
     {\small $4>0\leadsto$ \pawn{A}}
  \end{column}

  \begin{column}{.15\linewidth}\center
    \maisonCinq{1/A,2/C, 6/B, 9/A,10/C, 13/D,14/D, 17/B,18/E}
      {\flecherond{2}\maisonOk{4,5}}\\
     {\small $2>0\leadsto$ \pawn{C}}
  \end{column}

  \begin{column}{.15\linewidth}\center
    \maisonCinq{1/A, 5/C,6/B, 9/A,10/C, 13/D,14/D, 17/B,18/E}
      {\flecherond{1}\maisonOk{4,5}}\\
     {\small $2>0\leadsto$ \pawn{B}}
  \end{column}

  \begin{column}{.15\linewidth}\center
    \maisonCinq{1/A,2/B, 5/C,6/B, 9/A,10/C, 13/D,14/D, 18/E}
      {\flecherond{0}\maisonOk{4,5}}\\
     {\small $0=0\leadsto$ \pawn{D}}
  \end{column}

  \begin{column}{.15\linewidth}\center
    \maisonCinq{1/A,2/B, 5/C,6/B, 9/A,10/C, 14/D, 17/D,18/E}
      {\flecherond{4}\maisonOk{5}}\\
     {\small $3>0\leadsto$ \pawn{A}}
  \end{column}

  \begin{column}{.15\linewidth}\center
    \maisonCinq{1/A,2/B, 5/C,6/B, 10/C, 13/A,14/D, 17/D,18/E}
      {\flecherond{3}\maisonOk{5}}\\
     {\small $1>0\leadsto$ \pawn{C}}
  \end{column}
\end{columns}
%%%%%%%%%%%%%
\begin{columns}
  \begin{column}{.15\linewidth}\center
    \maisonCinq{1/A,2/B, 6/B, 9/C,10/C, 13/A,14/D, 17/D,18/E}
      {\flecherond{2}\maisonOk{3,5}}\\
     {\small $1>0\leadsto$ \pawn{B}}
  \end{column}

  \begin{column}{.15\linewidth}\center
    \maisonCinq{1/A, 5/B,6/B, 9/C,10/C, 13/A,14/D, 17/D,18/E}
      {\flecherond{1}\maisonOk{2,3,5}}\\
     {\small $4>0\leadsto$ \pawn{D}}
  \end{column}

  \begin{column}{.15\linewidth}\center
    \maisonCinq{1/A,2/D, 5/B,6/B, 9/C,10/C, 13/A,14/D, 18/E}
      {\flecherond{1}\maisonOk{2,3,5}}\\
     {\small On boucle!}
  \end{column}

  \begin{column}{.51\linewidth}
    \begin{itemize}
    \item On est revenu à notre situation initiale\\
      $\Rightarrow$ on va boucler à l'infini
    \item Cet algorithme ne marche pas ici, \\il faut chercher autre chose\ldots
    \end{itemize}
  \end{column}

\end{columns}

  \end{block}
\end{frame}
%%%%%%%%%%%%%%%%%%%%%%%%%%%%%%%%%%%%%%%%%%%%%%%%%%%%%%%%%%%%%%%%%%%%%%%%%%%%%%%%%%%%%%%%% 
\newcommand{\ligneMaison}[2]{
  \begin{tikzpicture}
    \foreach \x/\col in {#1} {
      \draw (.55*\x,.65) node[shape=circle,fill=\col,draw=black,inner sep=.1] {~~~};
    }
    \foreach \x/\col in {#2} {
      \draw (.55*\x,.3) node[shape=circle,fill=\col,draw=black,inner sep=.1] {~~~};
    }
    \foreach \x/\col in {1/A, 2/B, 3/C, 4/D, 5/E} {
      \draw (.55*\x,-.25) node[shape=ellipse,fill=\col,draw=black,rotate=90] {~~};
    }
  \end{tikzpicture}
}
%%%%%%%%%%%%%%%%%%%%%%%%%%%%%%%%%%%%%%%%%%%%%%%%%%%%%%%%%%%%%%%%%%%%%%%%%%%%%%%%%%%%%%%%%
\begin{frame}{Un autre algorithme pour le base-ball multicolore}
  \begin{block}{Apprendre de ses échecs: {\color{black}notre algorithme boucle parfois à l'infini}}
    \begin{itemize}\vspace{-.2\baselineskip}
    \item Pour réparer cela, le plus simple est de s'interdire de boucler, en
      coupant le cercle
    \item Les bonhommes ne peuvent plus passer directement de la maison
      \pawn{A} à \pawn{E}\\
      (pour ne pas se tromper, le plus simple est de placer les maisons en ligne)
%    \item Les règles de déplacement restent inchangées par ailleurs
    \end{itemize}

    \begin{columns}
      \begin{column}{.2\linewidth}
        ~
      \end{column}
      \begin{column}{.2\linewidth}\center
        \maisonCinq{1/D,2/B, 5/A,6/C, 9/E, 12/D,13/A, 17/B,18/C}{
          \draw[dashed,draw=red,ultra thick] (0,0) -- (.9,1.1);
        }         
      \end{column}

      \begin{column}{.1\linewidth}\center
        \Large$\leadsto$
      \end{column}

      \begin{column}{.2\linewidth}\center
        \ligneMaison{1/B, 2/C,      4/D, 5/B}
                    {1/D, 2/A, 3/E, 4/A, 5/C}
      \end{column}
      \begin{column}{.2\linewidth}
        ~
      \end{column}
    \end{columns}
  \end{block}

  \bigskip

  \begin{block}{Apprendre de ses réussites: {\color{black}le crépier}}
    \begin{itemize}\vspace{-.2\baselineskip}
    \item On a cherché à réduire la taille du problème à peu à peu\\
      {(il y a 7 crèpes à trier. La plus grande va définitivement à sa place; il
        reste 6 crèpes à trier)}
    \item On s'est fixé des objectifs intermédiaires, qui décomposent le
      problème en étapes que je sais faire\\
      {(mettre la plus grande en haut pour parvenir à la mettre en bas)}
    \end{itemize}
  \end{block}

  \begin{block}{Nouvel algorithme}
    \begin{itemize}\vspace{-.2\baselineskip}
    \item On place les bonshommes de la première maison en premier, et on n'y
      touche plus ensuite
    \item On fera ensuite pareil pour la seconde maison, puis la troisième, et
      ainsi de suite pour toutes
    \item Pour ammener les pions \pawn{A} dans leur maison, je déplace tous
      ceux qui me gènent
    \item Pour déplacer ceux qui me gènent, je déplace le trou pour leur faire de
      la place
    \end{itemize}
  \end{block}
  \begin{columns}
  % \begin{column}{.15\linewidth}\center
  %     \ligneMaison{1/B, 2/C, 3/D, 4/B}
  %                 {1/D, 2/A, 3/E, 4/A, 5/C}
  % \end{column}

  % \begin{column}{.15\linewidth}\center
  %     \ligneMaison{1/B, 2/C, 3/D,      5/B}
  %                 {1/D, 2/A, 3/E, 4/A, 5/C}
  % \end{column}
  \begin{column}{.15\linewidth}\center
      \ligneMaison{1/B, 2/C,      4/D, 5/B}
                  {1/D, 2/A, 3/E, 4/A, 5/C}
  \end{column}
  \begin{column}{.15\linewidth}\center
      \ligneMaison{1/B,      3/C, 4/D, 5/B}
                  {1/D, 2/A, 3/E, 4/A, 5/C}
  \end{column}
  \begin{column}{.15\linewidth}\center
      \ligneMaison{     2/B, 3/C, 4/D, 5/B}
                  {1/D, 2/A, 3/E, 4/A, 5/C}
  \end{column}
  \begin{column}{.15\linewidth}\center
      \ligneMaison{1/A, 2/B, 3/C, 4/D, 5/B}
                  {1/D,      3/E, 4/A, 5/C}
  \end{column}
  \begin{column}{.15\linewidth}\center
      \ligneMaison{1/A, 2/B, 3/C, 4/D, 5/B}
                  {1/D, 2/E,      4/A, 5/C}
  \end{column}    
  \begin{column}{.15\linewidth}\center
      \ligneMaison{1/A, 2/B, 3/C, 4/D, 5/B}
                  {1/D, 2/E, 3/A,      5/C}
  \end{column}    
\end{columns}\vspace{-\baselineskip}
\begin{columns}
  \begin{column}{.15\linewidth}\center
      \ligneMaison{1/A, 2/B,      4/D, 5/B}
                  {1/D, 2/E, 3/A, 4/C, 5/C}
  \end{column}    
  \begin{column}{.15\linewidth}\center
      \ligneMaison{1/A,      3/B, 4/D, 5/B}
                  {1/D, 2/E, 3/A, 4/C, 5/C}
  \end{column}    
  \begin{column}{.15\linewidth}\center
      \ligneMaison{1/A, 2/A, 3/B, 4/D, 5/B}
                  {1/D, 2/E,      4/C, 5/C}
  \end{column}    
  \begin{column}{.15\linewidth}\center
      \ligneMaison{1/A, 2/A, 3/B, 4/D, 5/B}
                  {1/D,      3/E, 4/C, 5/C}
  \end{column}    
  \begin{column}{.15\linewidth}\center
      \ligneMaison{1/A, 2/A, 3/B, 4/D, 5/B}
                  {     2/D, 3/E, 4/C, 5/C}
  \end{column}    
  \begin{column}{.15\linewidth}\center
      \ligneMaison{1/A,      3/B, 4/D, 5/B}
                  {1/A, 2/D, 3/E, 4/C, 5/C}
  \end{column}    
\end{columns}


  \begin{itemize}
  \item On peut maintenant oublier les \pawn{A}, qui sont à leur place définitive.
  \item On place maintenant les \pawn{B} de la même manière, puis les \pawn{C},
    puis les \pawn{D} et enfin le \pawn{E}
  \end{itemize}
\end{frame}
%%%%%%%%%%%%%%%%%%%%%%%%%%%%%%%%%%%%%%%%%%%%%%%%%%%%%%%%%%%%%%%%%%%%%%%%%%%%%%%%%%%%%%%%%
\end{Coupe}
\begin{frame}{Ce qu'il faut retenir du base-ball multicolore: corrections d'algorithmes}

  Cet algorithme n'est pas tellement plus complexe ou plus long que le
  précédent, et il est correct, lui.

  \begin{block}{Comment être sûr de la \alert{correction} de cet algorithme?}
    \begin{itemize}
    \item \structure{On pourrait tester tous les cas}. Fastidieux, mais les
      ordinateurs sont rapides; cela semble faisable jusqu'à une vingtaine ou
      une cinquantaine de maisons (mais ça reste partiel)
    \item \structure{On pourrait écrire une preuve mathématique}. Pas trivial
      du tout, mais les chercheurs en informatique en ont écrite des plus
      difficiles (c'est vite des maths un peu touffues).
    \item \structure{Cela ressemble vraiment à un algorithme classique} (même
      si cela ne prouve rien, au fond).
    \end{itemize}
  \end{block}

  \begin{block}{Qu'est ce qu'un \alert{algorithme classique}?}
    \begin{itemize}
    \item Effectivement, les informaticiens apprennent par cœur des algorithmes
      (abstraits) à l'école.
    \item Face à un problème nouveau, on cherche à se raccrocher à des
      problèmes connus
      \begin{itemize}
      \item On se raccroche en trouvant des analogies ou en décomposant en
        plusieurs problèmes connus
      \item Quand des collègues informaticiens jouent au crêpier, ils demandent
        avant tout si c'est "une tour de Hanoï" 
      \end{itemize}
    \item Ici, c'est assez proche d'un "tri à bulle", même si c'est pas
      absolument identique.
    \end{itemize}
  \end{block}

  \begin{block}{Les algorithmes de tri sont ultra classiques en informatique}
    \begin{itemize}
    \item Ils sont assez simple pour expliquer les grandes lignes aux élèves\\
      (comme «diviser pour régner» et autres grandes idées similaires --
      récursion, algo gloutons, \ldots)
    \item Les ordinateurs trient très souvent des données, car
      beaucoup de problèmes sont plus simples après\\
      (trouver un livre donné est plus simple dans une bibliothèque rangée, par exemple)
    \item Du coup, au chapitre 2 de mon cours d'algorithmique, on apprend 5
      algorithmes de tri par cœur: Tri à bulle, tri par sélection, tri par
      insertion, tri fusion, tri rapide (et quinze autres en exos).
    \item \alert{Les musiciens font leurs gammes, les informaticiens débutants
      apprennent leurs algorithmes}
    \end{itemize}
  \end{block}

  \begin{block}{Que font les chercheurs en informatique?}
    \begin{itemize}
    \item Certains d'entre eux améliorent les algorithmes connus, ou en
      inventent de nouveaux
    \item Il faut également démontrer la correction de ces algorithmes
    \item Quand plusieurs algorithmes existent, on étudie leurs \alert{performances} respectives
    \item (d'autres chercheurs améliorent matériel et logiciel, établissent des
      modèles, etc)
    \end{itemize}
  \end{block}
\end{frame}
%%%%%%%%%%%%%%%%%%%%%%%%%%%%%%%%%%%%%%%%%%%%%%%%%%%%%%%%%%%%%%%%%%%%%%%%%%%%%%%%%%%%%%%%%
\begin{frame}{Ce qu'il faut retenir du base-ball multicolore: performance d'algorithmes}
  \begin{block}{Comment comparer la performance des algorithmes?}
    \begin{itemize}
    \item Simplement en comptant les étapes. Par exemple sur le crépier, placer
      la grande crêpe prend 3 coups au pire. Et c'est pareil pour les
      suivantes, donc au pire, en $3\times n$ coups, j'ai tout trié
    \item La performance de mon algo dépend beaucoup de la situation initiale:
      \begin{itemize}
      \item Si c'est déjà trié, c'est de la chance, je n'ai rien à faire
        (\alert{meilleur cas}).
      \item Si j'ai vraiment pas de bol, il faut que je fasse les 3 étapes par
        crêpes (\alert{pire cas}).
      \item En pratique, j'ai souvent quelques crêpes qui arrivent déjà face
        cramée en haut, par exemple (\alert{cas moyen}).
      \end{itemize}
      Il faut bien comprendre que ceci ne dépend pas vraiment de l'algorithme,
      mais plutôt de la situation initiale. Le pire cas n'est pas un bug de
      l'algorithme, mais une situation initiale qui n'aide pas vraiment notre
      façon de faire (pour le cas moyen, il faut faire des probas).
    \item En pratique, une estimation de la performance est suffisante. Savoir
      que le crêpier nécessite environ $n^2$ étapes suffit, pas besoin de
      savoir que c'est $n^2+4$ ou $n^2-2$. Et même, $n^2$ ou $5\times
      n^2$, c'est un peu la même chose\ldots  On le note $O(n^2)$      
    \end{itemize}
  \end{block}

  \begin{block}{À la recherche du meilleur algorithme possible}
    \begin{itemize}
    \item On arrive parfois à montrer qu'on a le meilleur algorithme
      possible. Par exemple on ne peut pas trier les éléments en moins de $n$
      étapes, car on doit forcément tous les considérer.
    \item On peut aussi prouver qu'un tri comparatif ne peut pas se faire en
      moins de $n\times log(n)$ étapes, car il n'accumule pas assez
      d'information pour choisir la bonne permutation en moins d'étapes.
    \item Et puis la plupart du temps, on ne sait pas prouver que l'algorithme
      connu est le meilleur possible.  C'est alors le meilleur \textit{connu},
      sans être forcément le meilleur \textit{possible}.
    \end{itemize}
  \end{block}

  \begin{block}{À la recherche de problèmes difficiles}
    \begin{itemize}
    \item On peut classifier les problèmes en fonction de la performance des
      algorithmes les résolvant. \\
      (cela permet de se forger un sens commun de ce qui est faisable avec un
      ordinateur et éviter les problèmes si difficiles qu'ils sont quasi impossibles)
    \item Il existe énormément de problèmes relativement simples pour lesquels
      personne ne connaît de bon algorithme, sans que personne n'arrive non plus à
      démontrer qu'un tel algorithme n'existe pas.
    \item L'activité suivante sera l'occasion d'explorer un peu cette
      classification des problèmes très durs.
    \end{itemize}
  \end{block}
\end{frame}

%%% Local Variables: 
%%% mode: latex
%%% TeX-master: "CSIRL"
%%% End: 
