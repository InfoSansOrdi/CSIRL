\documentclass[10pt,a4paper]{article}

\usepackage[left=3cm,right=3cm,bottom=3cm]{geometry}
%,top=2cm,bottom=2cm
\usepackage[utf8]{inputenc}
\usepackage[T1]{fontenc}
\usepackage{lmodern} 
\usepackage[french]{babel}
\usepackage{tikz}
\usepackage{color} %for todos

\usepackage{hyperref}

\newcommand{\todo}[1]{\textcolor{red}{[#1]}}
\newcommand{\ttodo}[1]{\textcolor{red}{[TODO: #1]}}
\newcommand{\question}[2]{\noindent \textcolor{red}{#1}\\ #2}

\title{Activité réseau} 

\author{Annie Gravey (Télécom Bretagne / IRISA)}
\date{}
\pagestyle{empty}

\begin{document}
\maketitle
\thispagestyle{empty}

\section{Phase 1 : explication de l'objectif}

Quand une source envoie un fichier à un destinataire, la source doit être certaine que le fichier est bien reçu. Or il peut y avoir des erreurs dans le réseau, qui peut perdre des informations.

\textcolor{red}{Distribuer des localisations (source= Brest) et des
  destinations possibles (dont Rennes et Pekin par exemple)\\
  Donner à Brest un paquet bleu.\\
  Lui demander de le faire parvenir à Rennes\\
  QUESTION : comment Brest peut être certaine que Rennes a bien reçu le paquet ?\\ 
  On peut guider les élèves : Parler de lettre recommandée avec accusé de réception}

Donner un accusé de réception au récepteur


\section{Phase 2 : la notion d'accusé de réception}

Voler soit un paquet, soit un accusé de réception
L’accusé de réception peut aussi se perdre !

\textcolor{red}{QUESTION : que peut faire la source pour décider si le paquet
doit être retransmis ? Laisser les élèves s’exprimer. Il doit bien y en avoir
un qui pense à un temps limite d’attente ! sinon, leur parler d’un minuteur}

Donner le rôle de minuteur à un élève ; le faire se placer derrière la source
ou à côté (son voisin par exemple). Faire compter lentement (``Mississipi'') par
exemple jusqu’à 5. Faire marcher le système entre source et destination proches
avec un faible délai Peut se faire avec plusieurs paquets bleus non numérotés
et plusieurs accusés de réception non numérotés

Déplacer la destination beaucoup plus loin : 5 ne suffit plus !  Montrer que le
minuteur doit tenir compte du temps d’aller-retour entre Brest et la
destination.

\textit{On peut discuter des temps figurant sur les cartons de localisations
  Parler de la vitesse de la lumière : aller de la terre à Mars et retour (100
  millions de km : 300 secondes, pour un AR : 10 minutes !) de la vitesse du
  signal dans les câbles : un peu plus lente, 5s pour un million de km}


\section{Phase 3 : formalisation du protocole de transport en mode assuré}

\textcolor{red}{QUESTION: quel algorithme pour la destination ? quel algo pour
  la source ? Les élèves doivent proposer des algos qui permettent de conclure
cette première partie.}

\textcolor{red}{Algorithme pour la destination 
  \begin{itemize}
    \item LE RÉCEPTEUR ENVOIE un ACCUSÉ DE RÉCEPTION quand il reçoit un PAQUET
  \end{itemize}
  Algorithme pour la source 
  \begin{itemize}
    \item 1. QUAND LA SOURCE ENVOIE UN PAQUET, ELLE LANCE SON MINUTEUR
    \item 2. SI LA SOURCE REÇOIT L’ACCUSÉ DE RÉCEPTION AVANT L’EXPIRATION DU
      MINUTEUR, C’EST TERMINÉ, LE PAQUET PEUT ÊTRE EFFACÉ PAR LA SOURCE
    \item 3. SI LE MINUTEUR EXPIRE AVANT RÉCEPTION DE L’ACCUSÉ DE RÉCEPTION, LA
      SOURCE REPART À L’ÉTAPE 1, RENVOIT LE PAQUET ET RELANCE SON MINUTEUR
  \end{itemize}
  Dire que quand une source et une destination mettent en œuvre chacun leur
  algorithme, il s’agit d’un PROTOCOLE.
}

\emph{On peut avoir la question suivante : que se passe-t-il si la source
  reçoit l’accusé de réception de la première émission après avoir retransmis
  le paquet ? si la source respecte son algorithme elle réalise le 2) : jette
  le paquet et considère que tout s’est bien passé. Mais elle risque alors
  recevoir plus tard un accusé de réception sur un paquet qu’elle n’a pas gardé
  en mémoire. Dans la vraie vie, les messages sont identifiés, et la
  destination ne considère pas les accusés de réception de paquets qu’elle
n’attend pas.}

\section{Phase 3 : une transmission efficace}


Un fichier peut être tout petit ou très gros !  Le réseau n’accepte pas de
transporter en une fois un gros fichier. Un gros fichier est découpé en de
multiples paquets qui sont numérotés et doivent être tous reçus pour
reconstruire le fichier.


Donner les paquets numérotés à la source, les accusés numérotés à la
destination qui doit rester lointaine. 

Les faire envoyer et recevoir les paquets avec la règle suivante

\textcolor{red}{DONNER LES RÈGLES : 
  \begin{itemize}
    \item LE RÉCEPTEUR ENVOIE L’ACCUSÉ DE RÉCEPTION K S’IL A TOUT REÇU LE
      PAQUET K ; 
    \item LA SOURCE NE JETTE LE PAQUET K QUE S’IL A REÇU L’ACCUSÉ DE K ; IL
      PEUT ALORS ENVOYER K+1.
  \end{itemize}
}

C’est très lent !

\textcolor{red}{QUESTION : QUE PEUT ON FAIRE POUR ACCÉLERER LE TRANSFERT ?\\
LES RÈGLES COMPLÈTES SONT COMPLIQUÉES, SURTOUT POUR LA SOURCE !\\
Algorithme pour la destination :
\begin{itemize}
  \item LE RÉCEPTEUR ENVOIE L’ACCUSÉ DE RÉCEPTION K S’IL A TOUT REÇU TOUS LES
    PAQUETS JUSQU’AU PAQUET K ; S’IL REÇOIT K+2 AU LIEU DE K+1, IL ENVOIE DE
    NOUVEAU L’ACCUSÉ DE RÉCEPTION K, MAIS GARDE EN MÉMOIRE TOUS LES PAQUETS
    REÇUS.a
\end{itemize}
Algorithme pour la source :
\begin{itemize}
    \item LA SOURCE DISPOSE D’UNE FENETRE : ELLE PEUT PRENDRE DE L’AVANCE PAR
      RAPPORT AUX ACCUSÉS DE RÉCEPTION. SUPPOSONS QUE LA FENÊTRE SOIT DE TAILLE
      2
    \item QUAND LA SOURCE ENVOIE LE PAQUET K, ELLE LANCE SON MINUTEUR ; ELLE
      PEUT AUSSI ENVOYER LE PAQUET K+1 EN LANÇANT UN AUTRE MINUTEUR (LE
      MINUTEUR EST SPÉCIFIQUE AU PAQUET)
    \item SI LA SOURCE REÇOIT L’ACCUSÉ DE RÉCEPTION DE K AVANT L’EXPIRATION DU
      MINUTEUR K, LE PAQUET K PEUT ÊTRE EFFACÉ PAR LA SOURCE, ET LA SOURCE PEUT
      ENVOYER LE PAQUET K+2
    \item SI LA SOURCE REÇOIT L’ACCUSÉ DE RÉCEPTION DE K+1 AVANT L’EXPIRATION
      DU MINUTEUR, LES PAQUETS K ET K+1 PEUVENT ÊTRE EFFACÉS PAR LA SOURCE, ET
      LA SOURCE PEUT ENVOYER LES PAQUETS K+2 ET K+3
    \item SI LE MINUTEUR D’UN PAQUET EXPIRE AVANT RÉCEPTION DE L’ACCUSÉ DE
      RÉCEPTION DE CE PAQUET, LA SOURCE RENVOIT LE PAQUET ET RELANCE LE
      MINUTEUR DE CE PAQUET
    \item SI LA SOURCE REÇOIT UN ACCUSÉ DE RÉCEPTION D’UN PAQUET INFÉRIEUR À K,
      ELLE NE LE CONSIDÈRE PAS
  \end{itemize}
  Version simplifiée pour la source :
  \begin{itemize}
    \item POUR UNE FENETRE DE 2, la source PEUT ENVOYER K+1 ET K+2 Si elle A
      RECU L’ACCUSÉ DE RECEPTION K. LA SOURCE NE JETTE LE PAQUET K QUE S’IL
      elle a REÇU ACCUSÉ DE K OU AU DESSUS ; 
  \end{itemize}
}

Commencer avec une fenêtre de 2 paquets. Trouver un second minuteur !\\
Faire tourner le protocole, 
\begin{itemize}
  \item en volant par exemple 2 fois de suite le même paquet ;
  \item en volant seulement un accusé de réception.
\end{itemize}


\section*{Conclusion}
\begin{itemize}
  \item La source a mis en œuvre un algorithme, compliqué ! les minuteurs, les
    paquets qu’on garde en copie…
  \item La destination a aussi mis en œuvre un autre algorithme, plus simple
  \item Ensemble, ils ont fait marcher un PROTOCOLE
\end{itemize}


\emph{L’Internet, le téléphone, la télévision sur box, les RESEAUX en général, tout
cela marche avec des protocoles. Celui que l’on a vu ici est le plus répandu,
qui permet de garantir que tout fichier envoyé peut être bien reçu, même si
cela prend du temps !}

\section*{Questions possibles}

\question{Comment le paquet connaît il son chemin ?}{ce n’est pas le paquet qui
  connaît le chemin, ce sont les machines par oû transitent les paquets. Chaque
  paquet indique sa SOURCE et sa DESTINATION. Dans des réseaux simples, on peut
  mettre les machines en anneau : on arrivera toujours à son destinataire !
  dans un réseau plus compliqué, les machines sont arrangées comme sur une
  toile d’araignée et il faut d’autres PROTOCOLES pour apprendre où envoyer un
paquet, en fonction de sa destination }

\question{Les fichiers sont-ils gros dans l’Internet ?}{Il y a énormément de
  tout petits fichiers qui peuvent être portés dans un seul paquet. On parle de
  souris. Il y a aussi des gros fichiers : des videos par exemple. On parle
  alors d’éléphants. L’Internet peut porter à la fois des souris et des
  éléphants parce que tous les éléphants ont été découpés en un grand nombre de
paquets.}

\question{Comment sait-on combien de temps mettre sur le minuteur ?}{
  \begin{itemize}
    \item On a vu que la durée du minuteur dépendait de la distance ;
    \item On ne peut pas se mettre d’accord avant ! car justement on n’a pas
      commencé à parler à sa destination ;
    \item En fait, au début, il y a une valeur très grande (3s) (\emph{on peut
      revenir sur les distances données sur les cartons}) et au fur et à mesure,
      la source calcule le temps d’aller-retour et prend une durée plus grande
      que ce temps moyen.
  \end{itemize}
  }

\question{Qu’est ce qui fait qu’un même fichier peut arriver vite chez moi et
lentement chez mon voisin ?}{Cela peut dépendre de mon accès Internet ! Par
  exemple, considérons le contenu d’un DVD (environ 40Gbits) :
\begin{itemize}
  \item Sur un accès fibre très rapide, le fichier peut arriver en 40 secondes
    ;
  \item Sur un accès numéricable, il mettra déjà 10 fois plus de temps, près de
    7 minutes ;
  \item Sur un accès DSL, ce serait plutôt une heure, ou plus.
\end{itemize}
}

\question{Pourquoi cela prend parfois plus de temps de charger un fichier depuis chez
mon voisin que depuis les USA ?}{En fait, la durée n’est pas seulement fonction
  de la distance, mais aussi des caractéristiques du serveur !  Par exemple,
  sur une bretelle d’autoroute, le péage n’a que 2 barrières, il ne peut pas
  servir beaucoup de voitures par heure. Sur un gros péage, il peut y avoir
  jusqu’à 20 barrières, qui permettent de servir plus de voitures par heure. Le
  petit péage, c’est l’ordinateur du voisin, le gros péage, un serveur YouTube.
20 barrières de péage sont suffisantes sauf les jours de grand départ. Les
péages sont CONGESTIONNES ; les réseaux peuvent aussi l’être !}


\question{Est-ce que ce protocole est utilisé ?}{Sur L’INTERNET il est très
utilisé (vente sur Internet, transmission de photos, de documents
administratifs…) ; dans certains cas, on ne l’utilise pas car cela ralentirait
trop les échanges : par exemple pour le téléphone sur Internet, ce n’est pas
utilisé, ni pour la TV sur box (on voit parfois des images figées, ou des
petits carrés verts : certains paquets ont été perdus !).  Il y a d’autres
PROTOCOLES : par exemple avec un satellite, on enverra des accusés plus
compliqués, en disant explicitement ce qui n’a pas été reçu.}


\question{Pourquoi n’y a-t-il pas d’accusé de réception à l’accusé de réception
?}{Parce que l’accusé de réception n’est pas important, ce qui est important
c’est le paquet ! c’est la source qui fait les choses compliquées, le
destinataire a un algorithme beaucoup plus simple}


\question{Et un \emph{chat}, cela va dans les 2 sens !}{Et bien chacun est à la fois
source et destination}


\question{Quels sont les métiers des réseaux ?}{\begin{itemize}
    \item Tout le monde utilise les réseaux ;
    \item Toute l’informatique a besoin de réseaux ; par exemple les
      applications sur mobile
    \item Toutes les entreprises et les services publics ont besoin d’être sur
      Internet : une grosse structure a besoin d’un technicien ou ingénieur
      réseau pour tout faire marcher
    \item Il y a les entreprises qui vendent des services réseau : orange,
      numéricable, free \dots ;
    \item Il y a les fabricants d’équipements de réseau : Alcatel, Nokia,
      Thales (pour l’armée…).
  \end{itemize}
}

\question{Est ce qu’il y a encore des choses à inventer dans les réseaux ?}{Oui
  ! en 5 ans, entre 2014 et 2019, on prédit que le trafic aura triplé de volume
  ! Il faut des machines toujours plus puissantes, des tuyaux toujours plus
  gros, des méthodes toujours plus intelligentes pour supporter une telle
croissance. Analogie avec l’automobile : il y en avait déjà au 19ème siècle, et
il y a toujours de nouvelles inventions pour les perfectionner.}


\end{document}
